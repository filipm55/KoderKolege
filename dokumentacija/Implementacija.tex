\chapter{Implementacija i korisničko sučelje}
		
		
		\section{Korištene tehnologije i alati}
			
			U izradi dokumentacije i aplikacije koristili smo sljedeće tehnologije i alate:
			\begin{itemize}
			\item\textbf{pgAdmin} je open-source alat za upravljanje bazama podataka PostgreSQL. Koristi se za sve osnovne operacije na bazi podataka, kao što su kreiranje, ažuriranje i brisanje tablica, upita i drugih objekata. Također omogućuje napredne operacije, kao što su izvoz i uvoz podataka, upravljanje sigurnosnim ovlastima i optimizacija performansi. Više o pgAdminu možete saznati na https://www.pgadmin.org/.
			
			\item\textbf{React} je JavaScript biblioteka za izradu web sučelja (frontend razvoj). Koristi se za izgradnju dinamičnih i interaktivnih web stranica i aplikacija koje reagiraju na korisničke interakcije. React je jednostavan za učenje i korištenje, a također je vrlo fleksibilan.. Više o Reactu možete saznati na https://reactjs.org/.
			
			\item\textbf{Spring Boot} je open-source framework za Java backend razvoj. Koristi se za izgradnju RESTful web servisa i mikroservisa. Također se koristi za automatizaciju mnogih zadataka koji su inače potrebni za izgradnju web servisa, kao što su konfiguracija, upravljanje ovisnostima i sigurnost. Spring Boot je lagan i brz, a također je vrlo fleksibilan i prilagodljiv. Više o Spring Bootu možete saznati na https://spring.io/projects/spring-boot.
			
			\item\textbf{GitHub} je web platforma za upravljanje izvornim kodom. Koristi se za spremanje, dijeljenje i suradnju na izvornom kodu. Siguran je i pouzdan, a također vrlo fleksibilan.  Više o GitHubu možete saznati na https://github.com/.
			
			\item\textbf{WhatsApp} je mobilna aplikacija za razmjenu trenutnih poruka. Koristi se za komunikaciju unutar tima pomoću tekstualnih, glasovnih ili video poruka. WhatsApp je jednostavan za korištenje i pouzdan, a također je vrlo popularan Više o WhatsAppu možete saznati na https://www.whatsapp.com/.
			
			\item\textbf{Visual Paradigm Online} je online alat za izradu UML dijagrama. Koristi se za modeliranje softverskih sustava i drugih sustava. Visual Paradigm Online je jednostavan za korištenje i ima široku paletu funkcionalnosti. Više o Visual Paradigm Onlineu možete saznati na https://online.visual-paradigm.com/.
			
			\item\textbf{LateX} je programski jezik za pisanje strukturiranih tekstova i njihov automatski slog i prijelom u dokumente profesionalne kvalitete spremne za tisak. Omogućuje precizno kontroliranje izgleda dokumenta, uključujući veličinu i oblik slova, razmake, margine i druge elemente. LaTeX je popularan među znanstvenicima i inženjerima za pisanje tehničkih dokumenata. Više o LaTeXu možete saznati na https://www.latex-project.org/.
			\end{itemize}
			
			\textbf{Zaključak}\\
			U izradi dokumentacije i aplikacije korištene su suvremene tehnologije i alati koji su omogućili izgradnju kvalitetnog i funkcionalnog proizvoda.
			
			
		
	
		\section{Ispitivanje programskog rješenja}
			
			\textbf{\textit{dio 2. revizije}}\\
			
			 \textit{U ovom poglavlju je potrebno opisati provedbu ispitivanja implementiranih funkcionalnosti na razini komponenti i na razini cijelog sustava s prikazom odabranih ispitnih slučajeva. Studenti trebaju ispitati temeljnu funkcionalnost i rubne uvjete.}
	
			
			\subsection{Ispitivanje komponenti}
			\textit{Potrebno je provesti ispitivanje jedinica (engl. unit testing) nad razredima koji implementiraju temeljne funkcionalnosti. Razraditi \textbf{minimalno 6 ispitnih slučajeva} u kojima će se ispitati redovni slučajevi, rubni uvjeti te izazivanje pogreške (engl. exception throwing). Poželjno je stvoriti i ispitni slučaj koji koristi funkcionalnosti koje nisu implementirane. Potrebno je priložiti izvorni kôd svih ispitnih slučajeva te prikaz rezultata izvođenja ispita u razvojnom okruženju (prolaz/pad ispita). }
			
			
			
			\subsection{Ispitivanje sustava}
			
			 \textit{Potrebno je provesti i opisati ispitivanje sustava koristeći radni okvir Selenium\footnote{\url{https://www.seleniumhq.org/}}. Razraditi \textbf{minimalno 4 ispitna slučaja} u kojima će se ispitati redovni slučajevi, rubni uvjeti te poziv funkcionalnosti koja nije implementirana/izaziva pogrešku kako bi se vidjelo na koji način sustav reagira kada nešto nije u potpunosti ostvareno. Ispitni slučaj se treba sastojati od ulaza (npr. korisničko ime i lozinka), očekivanog izlaza ili rezultata, koraka ispitivanja i dobivenog izlaza ili rezultata.\\ }
			 
			 \textit{Izradu ispitnih slučajeva pomoću radnog okvira Selenium moguće je provesti pomoću jednog od sljedeća dva alata:}
			 \begin{itemize}
			 	\item \textit{dodatak za preglednik \textbf{Selenium IDE} - snimanje korisnikovih akcija radi automatskog ponavljanja ispita	}
			 	\item \textit{\textbf{Selenium WebDriver} - podrška za pisanje ispita u jezicima Java, C\#, PHP koristeći posebno programsko sučelje.}
			 \end{itemize}
		 	\textit{Detalji o korištenju alata Selenium bit će prikazani na posebnom predavanju tijekom semestra.}
			
			\eject 
		
		
		\section{Dijagram razmještaja}
			
			\textbf{\textit{dio 2. revizije}}
			
			 \textit{Potrebno je umetnuti \textbf{specifikacijski} dijagram razmještaja i opisati ga. Moguće je umjesto specifikacijskog dijagrama razmještaja umetnuti dijagram razmještaja instanci, pod uvjetom da taj dijagram bolje opisuje neki važniji dio sustava.}
			
			\eject 
		
		\section{Upute za puštanje u pogon}
 
			
			Postavljanje aplikacije u pogon putem usluge Render zahtijeva pažljivo provedene korake uključujući kreiranje baze podataka, kreiranje backend-a  te kreiranje frontenda, čime se osigurava siguran i učinkovit deploy.
			
			Prije samog puštanja aplikacije u pogon potrebno je provesti tri ključna koraka:
			\begin{packed_enum}
				\item \textbf{Konfiguracija baze podataka}
				\begin{itemize}
					\item Ovaj proces uključuje definiranje environmental varijabli unutar konfiguracije razvojnog okruženja (IDE). U našem slučaju, konfiguracija se nalazi u datoteci src/main/resources/application.properties. U ovoj datoteci specificiramo parametre kao što su korisničko ime, lozinka, URL baze podataka te određujemo port poslužitelja. Primjer konfiguracije:
				\end{itemize}
				
				\item \textbf{Priprema backend-a za deploy}
				\begin{itemize}
					\item Nakon postavljanja baze podataka, slijedi priprema backend-a za deploy. Ovaj korak uključuje dodavanje Dockerfile skripte za izgradnju i pokretanje aplikacije.  U našem slučaju, koristimo Maven, pa je Dockerfile konfiguriran na sljedeći način:
					\item Ukoliko se mijenja lokacija Dockerfilea paziti na putanje unutar COPY naredbi u Dockerfile skripti. U nasem slucaju Docker file se nalazi u root direktoriju Docker/Maven/Dockerfile te su po toj putanji pisane naredbe COPY.
				\end{itemize}
				
				\item \textbf{Priprema frontenda za deploy}
				\begin{itemize}
					\item Potrebno je dodati potrebne dependencije u package.json datoteku. U terminalu je potrebno izvršiti sljedeće naredbe:
					\item Potrebno je kreirati Proxy.js, u src direktorij, koji služi kao proxy server za lokalni development (preusmjeruje api pozive na localhost:8080). Ovo je primjer koda:
					\item Stvoriti app.js, u root direktorij, koja sadrži Express server za produkcijski proxy i posluživanje frontenda. Ovo je primjer koda:
					\item Izmjeniti package.json skripte i dodati specifične konfiguracije:
				\end{itemize}
			\end{packed_enum}
			
			Sada, nakon završene pripreme, možemo započeti deploy.
			
				\begin{packed_enum}
				\item \textbf{Kreiranje baze podataka}
				\begin{itemize}
					\item
				\end{itemize}
				
				\item \textbf{Kreiranje backend-a}
				\begin{itemize}
					\item 
				\end{itemize}
				
				\item \textbf{Kreiranje frontenda}
				\begin{itemize}
					\item 
				\end{itemize}
			\end{packed_enum}
		
				
			
			
			