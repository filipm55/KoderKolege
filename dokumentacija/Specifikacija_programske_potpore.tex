\chapter{Specifikacija programske potpore}
		
		
\section{Funkcionalni zahtjevi}

\textbf{\textit{dio 1. revizije}}\\

\textit{Navesti \textbf{dionike} koji imaju \textbf{interes u ovom sustavu} ili  \textbf{su nositelji odgovornosti}. To su prije svega korisnici, ali i administratori sustava, naručitelji, razvojni tim.}\\

\textit{Navesti \textbf{aktore} koji izravno \textbf{koriste} ili \textbf{komuniciraju sa sustavom}. Oni mogu imati inicijatorsku ulogu, tj. započinju određene procese u sustavu ili samo sudioničku ulogu, tj. obavljaju određeni posao. Za svakog aktora navesti funkcionalne zahtjeve koji se na njega odnose.}\\


\noindent \textbf{Dionici:}

\begin{packed_enum}
	\item Naručitelj
	\item Voditelj natjecanja
	\item Natjecatelj				
	\item Administrator
	\item Razvojni tim
	
\end{packed_enum}

\noindent \textbf{Aktori i njihovi funkcionalni zahtjevi:}


\begin{packed_enum}
	\item  \underbar{Neregistrirani korisnik (inicijator) može:}
	
	\begin{packed_enum}
		
		\item pregledavati programske zadatke objavljene na stranici
		\item vidjeti kalendar s dostupnim natjecanjima
		\item pregledavati profile drugih korisnika (natjecatelja i vodietlja natjecanja)
		\begin{packed_enum}
			
			\item  na profilu natjecatelja može vidjeti statistike o broju točno riješenih zadataka, broju isprobanih zadataka i 
			pehare za osvojena natjecanja 
			\item  na profilu voditelja može vidjeti popis učitanih zadataka i kalendar s popisom objavljenih natjecanja
			
		\end{packed_enum}
		\item sortirati učitane zadatke na profilu voditelja natjecanja
		\item poslati zahtjev za registracijom za koji mora priložiti sljedeće informacije: uloga za koju se prijavljuje (voditelj natjecanja ili natjecatelj), korisničko ime,
		fotografija, lozinka, ime, prezime i email adresa
		
	\end{packed_enum}
	
	\item \underbar{Natjecatelj (inicijator) može:}
	\begin{packed_enum}
		
		\item za vrijeme trajanja natjecanja:
		\begin{packed_enum}
			\item vidjeti aktualne zadatke
			\item poslati datoteku s programskim kodom za svaki zadatak
		\end{packed_enum}
		\item nakon natjecanja:
		\begin{packed_enum}
			\item  vidjeti popis učitanih rješenja drugih natjecatelja
			\item za svaki pojedini zadatak vidjeti popis svih natjecatelja koji su učitali rješenje za taj zadatak, broj točnih primjera po najboljem učitavanju od natjecatelja i
			prosječno vrijeme izvršavanja po primjeru (???sta ovo znaci u tekstu???)
			\item dohvatiti učitano rješenje za pojedini zadatak ukoliko je on u potpunosti točno riješen
		\end{packed_enum}
		
		\item vježbati prethodno objavljene zadatke 
		\begin{packed_enum}
			\item učitati rješenje zadatka u aplikaciju
		\end{packed_enum}
		
		\item pokrenuti virtualno natjecanje
		
	\end{packed_enum}
	
	\item \underbar{Voditelj natjecanja (inicijator) može:}
	\begin{packed_enum}
		
		\item učitati nove zadatke u aplikaciju
		\item organizirati natjecanje:
			\begin{packed_enum}
				\item odabire vrijeme početka i završetka
				\item odlučuje broj zadataka
				\item odlučuje koji su zadaci aktivni 
				\item po želji učitava sličicu pehara (??)
			\end{packed_enum}
		\item izraditi zadatak:
			\begin{packed_enum}
				\item unosi naziv zadatka
				\item unosi broj bodova (ovisan o težini zadatka)
				\item određuje vremensko ograničenje izvršavanja zadatka
				\item unosi tekst zadatka
				\item unosi testne primjere za evaluaciju (provjeravaju ulaz i izlaz programa)
				\item može zadatak postaviti kao privatan te on nakon završetka natjecanja automatski postaje javan
			\end{packed_enum}}
		\item uređivati vlastito objavljene zadatke i natjecanja (to ne mijenja prijašnje rezultate)

	\end{packed_enum}
	
	\item \underbar{Administrator (inicijator) može:}
	\begin{packed_enum}
		
		\item vidjeti popis svih registriranih korisnika i njihovih osobnih podataka
		\item mijenjati dodijeljena prava registriranim korisnicima
		\item mijenjati osobne podatke registriranih korisnika
		\item potvrditi/odbiti registracijski zahtjev za ulogu voditelja
		\item uređivati sve zadatke i natjecanja
		
	\end{packed_enum}
	
	\item  \underbar{Baza podataka (sudionik) može:}
	
	\begin{packed_enum}
		
		\item funkcionalnost 1
		\item funkcionalnost 2
		
	\end{packed_enum}
\end{packed_enum}

\eject 
			
				
			\subsection{Obrasci uporabe}
				
				\textbf{\textit{dio 1. revizije}}
				
				\subsubsection{Opis obrazaca uporabe}
					\textit{Funkcionalne zahtjeve razraditi u obliku obrazaca uporabe. Svaki obrazac je potrebno razraditi prema donjem predlošku. Ukoliko u nekom koraku može doći do odstupanja, potrebno je to odstupanje opisati i po mogućnosti ponuditi rješenje kojim bi se tijek obrasca vratio na osnovni tijek.}\\
					

					\noindent \underbar{\textbf{UC$<$broj obrasca$>$ -$<$ime obrasca$>$}}
					\begin{packed_item}
	
						\item \textbf{Glavni sudionik: }$<$sudionik$>$
						\item  \textbf{Cilj:} $<$cilj$>$
						\item  \textbf{Sudionici:} $<$sudionici$>$
						\item  \textbf{Preduvjet:} $<$preduvjet$>$
						\item  \textbf{Opis osnovnog tijeka:}
						
						\item[] \begin{packed_enum}
	
							\item $<$opis korak jedan$>$
							\item $<$opis korak dva$>$
							\item $<$opis korak tri$>$
							\item $<$opis korak četiri$>$
							\item $<$opis korak pet$>$
						\end{packed_enum}
						
						\item  \textbf{Opis mogućih odstupanja:}
						
						\item[] \begin{packed_item}
	
							\item[2.a] $<$opis mogućeg scenarija odstupanja u koraku 2$>$
							\item[] \begin{packed_enum}
								
								\item $<$opis rješenja mogućeg scenarija korak 1$>$
								\item $<$opis rješenja mogućeg scenarija korak 2$>$
								
							\end{packed_enum}
							\item[2.b] $<$opis mogućeg scenarija odstupanja u koraku 2$>$
							\item[3.a] $<$opis mogućeg scenarija odstupanja  u koraku 3$>$
							
						\end{packed_item}
					\end{packed_item}
				
					
				\subsubsection{Dijagrami obrazaca uporabe}
					
					\textit{Prikazati odnos aktora i obrazaca uporabe odgovarajućim UML dijagramom. Nije nužno nacrtati sve na jednom dijagramu. Modelirati po razinama apstrakcije i skupovima srodnih funkcionalnosti.}
				\eject		
				
			\subsection{Sekvencijski dijagrami}
				
				\textbf{\textit{dio 1. revizije}}\\
				
				\textit{Nacrtati sekvencijske dijagrame koji modeliraju najvažnije dijelove sustava (max. 4 dijagrama). Ukoliko postoji nedoumica oko odabira, razjasniti s asistentom. Uz svaki dijagram napisati detaljni opis dijagrama.}
				\eject
	
		\section{Ostali zahtjevi}
		
			\textbf{\textit{dio 1. revizije}}\\
		 
			 \textit{Nefunkcionalni zahtjevi i zahtjevi domene primjene dopunjuju funkcionalne zahtjeve. Oni opisuju \textbf{kako se sustav treba ponašati} i koja \textbf{ograničenja} treba poštivati (performanse, korisničko iskustvo, pouzdanost, standardi kvalitete, sigurnost...). Primjeri takvih zahtjeva u Vašem projektu mogu biti: podržani jezici korisničkog sučelja, vrijeme odziva, najveći mogući podržani broj korisnika, podržane web/mobilne platforme, razina zaštite (protokoli komunikacije, kriptiranje...)... Svaki takav zahtjev potrebno je navesti u jednoj ili dvije rečenice.}
			 
			 
			 
	