\chapter{Specifikacija programske potpore}
		
		
\section{Funkcionalni zahtjevi}

\textbf{\textit{dio 1. revizije}}\\

\textit{Navesti \textbf{dionike} koji imaju \textbf{interes u ovom sustavu} ili  \textbf{su nositelji odgovornosti}. To su prije svega korisnici, ali i administratori sustava, naručitelji, razvojni tim.}\\

\textit{Navesti \textbf{aktore} koji izravno \textbf{koriste} ili \textbf{komuniciraju sa sustavom}. Oni mogu imati inicijatorsku ulogu, tj. započinju određene procese u sustavu ili samo sudioničku ulogu, tj. obavljaju određeni posao. Za svakog aktora navesti funkcionalne zahtjeve koji se na njega odnose.}\\


\noindent \textbf{Dionici:}

\begin{packed_enum}
	\item Naručitelj
	\item Voditelj natjecanja
	\item Natjecatelj				
	\item Administrator
	\item Razvojni tim
	
\end{packed_enum}

\noindent \textbf{Aktori i njihovi funkcionalni zahtjevi:}


\begin{packed_enum}
	\item  \underbar{Neregistrirani korisnik (inicijator) može:}
	
	\begin{packed_enum}
		
		\item pregledavati programske zadatke objavljene na stranici
		\item vidjeti kalendar s dostupnim natjecanjima
		\item pregledavati profile drugih korisnika (natjecatelja i vodietlja natjecanja)
		\begin{packed_enum}
			
			\item  na profilu natjecatelja može vidjeti statistike o broju točno riješenih zadataka, broju isprobanih zadataka i 
			pehare za osvojena natjecanja 
			\item  na profilu voditelja može vidjeti popis učitanih zadataka i kalendar s popisom objavljenih natjecanja
			
		\end{packed_enum}
		\item sortirati učitane zadatke na profilu voditelja natjecanja
		\item poslati zahtjev za registracijom za koji mora priložiti sljedeće informacije: uloga za koju se prijavljuje (voditelj natjecanja ili natjecatelj), korisničko ime,
		fotografija, lozinka, ime, prezime i email adresa
		
	\end{packed_enum}
	
	\item \underbar{Natjecatelj (inicijator) može:}
	\begin{packed_enum}
		
		\item za vrijeme trajanja natjecanja:
		\begin{packed_enum}
			\item vidjeti aktualne zadatke
			\item poslati datoteku s programskim kodom za svaki zadatak
		\end{packed_enum}
		\item nakon natjecanja:
		\begin{packed_enum}
			\item  vidjeti popis učitanih rješenja drugih natjecatelja
			\item za svaki pojedini zadatak vidjeti popis svih natjecatelja koji su učitali rješenje za taj zadatak, broj točnih primjera po najboljem učitavanju od natjecatelja i
			prosječno vrijeme izvršavanja po primjeru (???sta ovo znaci u tekstu???)
			\item dohvatiti učitano rješenje za pojedini zadatak ukoliko je on u potpunosti točno riješen
		\end{packed_enum}
		
		\item vježbati prethodno objavljene zadatke 
		\begin{packed_enum}
			\item učitati rješenje zadatka u aplikaciju
		\end{packed_enum}
		
		\item pokrenuti virtualno natjecanje
		
	\end{packed_enum}
	
	\item \underbar{Voditelj natjecanja (inicijator) može:}
	\begin{packed_enum}
		
		\item učitati nove zadatke u aplikaciju
		\item organizirati natjecanje:
			\begin{packed_enum}
				\item odabire vrijeme početka i završetka
				\item odlučuje broj zadataka
				\item odlučuje koji su zadaci aktivni 
				\item po želji učitava sličicu pehara (??)
			\end{packed_enum}
		\item izraditi zadatak:
			\begin{packed_enum}
				\item unosi naziv zadatka
				\item unosi broj bodova (ovisan o težini zadatka)
				\item određuje vremensko ograničenje izvršavanja zadatka
				\item unosi tekst zadatka
				\item unosi testne primjere za evaluaciju (provjeravaju ulaz i izlaz programa)
				\item može zadatak postaviti kao privatan te on nakon završetka natjecanja automatski postaje javan
			\end{packed_enum}
		\item uređivati vlastito objavljene zadatke i natjecanja (to ne mijenja prijašnje rezultate)

	\end{packed_enum}
	
	\item \underbar{Administrator (inicijator) može:}
	\begin{packed_enum}
		
		\item vidjeti popis svih registriranih korisnika i njihovih osobnih podataka
		\item mijenjati dodijeljena prava registriranim korisnicima
		\item mijenjati osobne podatke registriranih korisnika
		\item potvrditi/odbiti registracijski zahtjev za ulogu voditelja
		\item uređivati sve zadatke i natjecanja
		
	\end{packed_enum}
	
	\item  \underbar{Baza podataka (sudionik) može:}
	
	\begin{packed_enum}
		
		\item funkcionalnost 1
		\item funkcionalnost 2
		
	\end{packed_enum}
\end{packed_enum}

\eject 
			
				
			\subsection{Obrasci uporabe}
				
				\textbf{\textit{dio 1. revizije}}
				
				\subsubsection{Opis obrazaca uporabe}
					\textit{Funkcionalne zahtjeve razraditi u obliku obrazaca uporabe. Svaki obrazac je potrebno razraditi prema donjem predlošku. Ukoliko u nekom koraku može doći do odstupanja, potrebno je to odstupanje opisati i po mogućnosti ponuditi rješenje kojim bi se tijek obrasca vratio na osnovni tijek.}\\
					

					\noindent \underbar{\textbf{UC$<$broj obrasca$>$ -$<$ime obrasca$>$}}
					\begin{packed_item}
	
						\item \textbf{Glavni sudionik: }$<$sudionik$>$
						\item  \textbf{Cilj:} $<$cilj$>$
						\item  \textbf{Sudionici:} $<$sudionici$>$
						\item  \textbf{Preduvjet:} $<$preduvjet$>$
						\item  \textbf{Opis osnovnog tijeka:}
						
						\item[] \begin{packed_enum}
	
							\item $<$opis korak jedan$>$
							\item $<$opis korak dva$>$
							\item $<$opis korak tri$>$
							\item $<$opis korak četiri$>$
							\item $<$opis korak pet$>$
						\end{packed_enum}
						
						\item  \textbf{Opis mogućih odstupanja:}
						
						\item[] \begin{packed_item}
	
							\item[2.a] $<$opis mogućeg scenarija odstupanja u koraku 2$>$
							\item[] \begin{packed_enum}
								
								\item $<$opis rješenja mogućeg scenarija korak 1$>$
								\item $<$opis rješenja mogućeg scenarija korak 2$>$
								
							\end{packed_enum}
							\item[2.b] $<$opis mogućeg scenarija odstupanja u koraku 2$>$
							\item[3.a] $<$opis mogućeg scenarija odstupanja  u koraku 3$>$
							
						\end{packed_item}
					\end{packed_item}
					
					
					
						\noindent \underbar{\textbf{UC - Izrada natjecanja}}
					\begin{packed_item}
						
						\item \textbf{Glavni sudionik: }Voditelj
						\item  \textbf{Cilj:} Napraviti natjecanje za korisnike 
						\item  \textbf{Sudionici:} Natjecatelji
						\item  \textbf{Preduvjet:} Registrirani korisnik voditelj
						\item  \textbf{Opis osnovnog tijeka:}
						
						\item[] \begin{packed_enum}
							
							\item Voditelj odabire opciju izrade natjecanja
							\item Odabire vrijeme početka i završetka natjecanja
							\item Bira broj zadataka 
							\item Unosi tekst zadatka
							\item Bira broj bodova za zadatak 
							\item Odabire po želji sličicu pehara 
						\end{packed_enum}
					\end{packed_item}
					
						\noindent \underbar{\textbf{UC Prijava u sustav}}
					\begin{packed_item}
						
						\item \textbf{Glavni sudionik: }Ghost
						\item  \textbf{Cilj:} Dobiti pristup korisničkom sustavu
						\item  \textbf{Sudionici:} Baza podataka
						\item  \textbf{Preduvjet:} Registracija
						\item  \textbf{Opis osnovnog tijeka:}
						
						\item[] \begin{packed_enum}
							
							\item Unos korisničkog imena i lozinke
							\item Potvrda od sustava o ispravnim podatcima
							\item Pristup korisničkim opcijama
						\end{packed_enum}
						
						\item  \textbf{Opis mogućih odstupanja:}
						
						\item[] \begin{packed_item}
							
							\item[2.a] Neispravno korisnicko ime ili lozinka
							\item[] \begin{packed_enum}
								
								\item Korisnik dobiva poruku o neispravnom korisnickom imenu ili lozinki
							\end{packed_enum}

						\end{packed_item}
					\end{packed_item}
					
					
					\noindent \underbar{\textbf{UC2 - Pregled kalendara}}
					\begin{packed_item}
						
						\item \textbf{Glavni sudionik: }Korisnik, Ghost
						\item  \textbf{Cilj:} Pregled kalendara sa svim natjecanjima 
						\item  \textbf{Sudionici:} Baza podataka
						\item  \textbf{Preduvjet:} /
						\item  \textbf{Opis osnovnog tijeka:}
						
						\item[] \begin{packed_enum}
							
							\item Na pocetnoj stranici je prikazan tjedni kalendar
							\item Korisnik/ghost odabire mjesečnu opciju prikaza kalendara
							\item Otvara se stranica sa prikazom mjesečnog kalendara natjecanja
							
						\end{packed_enum}
						
						\item  \textbf{Opis mogućih odstupanja:}
						
						\item[] \begin{packed_item}
							
							\item[3.a] Ghost želi pristupiti natjecanju
							\item[] \begin{packed_enum}
								
								\item Korisnik dobiva poruku da je za nastavak potrebna registracija 

							\end{packed_enum}

							
						\end{packed_item}
					\end{packed_item}
					
						\noindent \underbar{\textbf{UC1 - Pregled zadataka}}
					\begin{packed_item}
						
						\item \textbf{Glavni sudionik: }Korisnik, Ghost
						\item  \textbf{Cilj:} Pregled svih zadataka u aplikaciji
						\item  \textbf{Sudionici:} Baza podataka
						\item  \textbf{Preduvjet:} /
						\item  \textbf{Opis osnovnog tijeka:}
						
						\item[] \begin{packed_enum}
							
							\item Na pocetnoj stranici korisnik/ghost odabire opciju pregled svih zadataka
							\item Otvara se stranica s popisom svih zadataka
							\item Korisnik/ghost odabire zadatak
							\item Otvara se stranica sa zadatkom
						\end{packed_enum}
						
						\item  \textbf{Opis mogućih odstupanja:}
						
						\item[] \begin{packed_item}
							
							\item[4.a] Ghost (neregistrirani korisnik) želi predati riješenje zadatka 
							\item[] \begin{packed_enum}
								
								\item Korisnik dobiva poruku da je za nastavak potrebna registracija
								
							\end{packed_enum}
							
						\end{packed_item}
					\end{packed_item}
					
						\noindent \underbar{\textbf{UC - Registracija}}
					\begin{packed_item}
						
						\item \textbf{Glavni sudionik: } Ghost
						\item  \textbf{Cilj:} Registracija novog korisnika
						\item  \textbf{Sudionici:} Baza podataka
						\item  \textbf{Preduvjet:}  / 
						\item  \textbf{Opis osnovnog tijeka:}
						
						\item[] \begin{packed_enum}
							\item Korisnik odabire opciju za registraciju
							\item Otvara se form u koji upisuje podatke:
							\item[] \begin{packed_enum}
								
								\item korisničko ime
								\item fotografija
								\item lozinka
								\item ime i prezime
								\item email adresa
								\item odabire: voditelj natjecanja / natjecatelj
								
							\end{packed_enum}
							\item Upisuje i odabire potrebne podatke			
							\item Učitava se stranica na kojoj piše da korisnik treba potvrditi mail
							\item Korisnik potvrđuje registraciju
							\item Korisniku se šalje mail o uspješnoj registraciji
						\end{packed_enum}
						
						\item  \textbf{Opis mogućih odstupanja:}
						
						\item[] \begin{packed_item}
							
							\item[2.a]Email/korisničko ime su već zauzeti
							\item[] \begin{packed_enum}
								
								\item Korisnik dobiva poruku da je mail/korisničko ime već korišten
								\item Traži se ponovni upis podataka
								
							\end{packed_enum}
							\item[2.f]Korisnik se registrira kao "voditelj natjecanja"
							\item[] \begin{packed_enum}
								
								\item Administrator dobiva poruku u kojoj (ne)potvrđuje registraciju

								
							\end{packed_enum}
							\item[4.a] Korisnik ne potvrđuje email
							\item[] \begin{packed_enum}
								
								\item Korisnikov profil se ne aktivira
								
							\end{packed_enum}

							
						\end{packed_item}
					\end{packed_item}
					\noindent \underbar{\textbf{UC2 - Pregled i uređivanje korisnika}}
						\begin{packed_item}
						
						\item \textbf{Glavni sudionik: } Administrator
						\item  \textbf{Cilj:} Učikovita administracija i održavanje sustava
						\item  \textbf{Sudionici:} Baza podataka, administrator
						\item  \textbf{Preduvjet:}  / 
						\item  \textbf{Opis osnovnog tijeka:}
						
						\item[] \begin{packed_enum}
							\item Administrator odabire opciju za pregled svih reg. korisnika
							\item Otvara se stranica s popisom svih registriranih korisnika i njihovih podataka
							\item Administrator mijenja podatke i prava korisnika i sprema promijene			
							\item Korisniku se šalje mail o promijeni njegovih podataka.
						\end{packed_enum}
						
						\item  \textbf{Opis mogućih odstupanja:}
						
						\item[] \begin{packed_item}
							
							\item[3.a]Administrator narušava integritet jedinstvenosti (npr. mijenja korisničko ime u već postojeće)
							\item[] \begin{packed_enum}
								
								\item Administrator dobiva poruku da je korisničko ime već korišteno
								\item Traži se ponovni upis podataka
								
							\end{packed_enum}
								
							\end{packed_item}
							
						\end{packed_item}
						
					\noindent \underbar{\textbf{UC3 - Pregled i uređivanje svih natjecanja \ zadataka}}
						\begin{packed_item}
							
							\item \textbf{Glavni sudionik: } Administrator
							\item  \textbf{Cilj:} Ispravljanje grešaka ili nesporazuma u natjecanju
							\item  \textbf{Sudionici:} Baza podataka
							\item  \textbf{Preduvjet:}  / 
							\item  \textbf{Opis osnovnog tijeka:}
							
							\item[] \begin{packed_enum}
								\item Administrator odabire opciju za pregled svih natjecanja
								\item Otvara se stranica s popisom svih natjecanja
								\item Administrator odabire natjecanje		
								\item Otvaraju se podaci o natjecanju i popis zadataka tog natjecanja
								\item Administrator uređuje podatke o natjecanju ili zadatke tog natjecanja
								\item Administrator sprema promijene
							\end{packed_enum}
							
							\item  \textbf{Opis mogućih odstupanja: - } 
					
						\end{packed_item}
						
					\noindent \underbar{\textbf{UC4 - Rješavanje prethodno objavljenih zadataka}}
				\begin{packed_item}
					
					\item \textbf{Glavni sudionik: } Natjecatelj
					\item  \textbf{Cilj:} Vježbanje zadataka za natjecanje
					\item  \textbf{Sudionici:} Baza podataka
					\item  \textbf{Preduvjet:}  / 
					\item  \textbf{Opis osnovnog tijeka:}
					
					\item[] \begin{packed_enum}
						\item Natjecatelj odabire opciju za pregled svih zadataka
						\item Otvara se stranica s popisom svih zadataka
						\item Korisnik odabire zadatak	
						\item Otvara se stranica sa zadatkom
						\item Natjecatelj može pogledati riješenje zadatka		
						\item Nakon riješavanja, natjecatelj odabire opciju za evaluaciju riješenja
						\item Natjecatelj dobiva informaciju o točnosti svog riješenja
					\end{packed_enum}
					
					\item  \textbf{Opis mogućih odstupanja: - } 
					
					
				\end{packed_item}
				
			\noindent \underbar{\textbf{UC5 - Pokretanje virtualnog natjecanja}}
			\begin{packed_item}
				
				\item \textbf{Glavni sudionik: } Natjecatelj
				\item  \textbf{Cilj:} Vježbanje zadataka za natjecanje
				\item  \textbf{Sudionici:} Baza podataka
				\item  \textbf{Preduvjet:}  / 
				\item  \textbf{Opis osnovnog tijeka:}
				
				\item[] \begin{packed_enum}
					\item Natjecatelj odabire neko prošlo natjecanje u kalendaru ili odabire nasumične zadatke
					\item Otvara se stranica s natjecanjem / zadacima
					\item Natjecatelj riješava zadatke unutar vremenskog ograničenja
					\item Natjecatelj predaje riješenja zadataka
					\item Po isteku vremenskog ograničenja ili nakon predaje svih riješenja, otvara se nova stranica u kojoj se korisnika rangira u odnosu na službene rezultate.
				\end{packed_enum}
				\item  \textbf{Opis mogućih odstupanja:  } 
				\item[] \begin{packed_enum}
				
					\item[1.a]Odabirom nasumičnih zadataka, korisnik dobiva ravnomjerno raspoređene nasumične zadatke
				
				\end{packed_enum}
				\end{packed_item}
				
				\noindent \underbar{\textbf{UC6 - Sudjelovanje u natjecanju}}
			\begin{packed_item}
				
				\item \textbf{Glavni sudionik: } Natjecatelj
				\item  \textbf{Cilj:} Natjecanje
				\item  \textbf{Sudionici:} Baza podataka
				\item  \textbf{Preduvjet:}  / 
				\item  \textbf{Opis osnovnog tijeka:}
				
				\item[] \begin{packed_enum}
					\item Natjecatelj odabire opciju otvaranja natjecanja
					\item Otvara se stranica s natjecanjem
					\item U trenutku početka natjecanja, svi zadaci postaju vidljivi
					\item Natjecatelj riješava zadatke unutar vremenskog ograničenja
					\item Natjecatelj predaje riješenja zadataka
					\item Po isteku vremenskog ograničenja, korisnik može otvoriti novu stranicu u kojoj se korisnika rangira u odnosu na ostale natjecatelje.
				\end{packed_enum}
			\end{packed_item}
				
					
				\subsubsection{Dijagrami obrazaca uporabe}
					
					\textit{Prikazati odnos aktora i obrazaca uporabe odgovarajućim UML dijagramom. Nije nužno nacrtati sve na jednom dijagramu. Modelirati po razinama apstrakcije i skupovima srodnih funkcionalnosti.}
				\eject		
				
			\subsection{Sekvencijski dijagrami}
				
				\textbf{\textit{dio 1. revizije}}\\
				
				\textit{Nacrtati sekvencijske dijagrame koji modeliraju najvažnije dijelove sustava (max. 4 dijagrama). Ukoliko postoji nedoumica oko odabira, razjasniti s asistentom. Uz svaki dijagram napisati detaljni opis dijagrama.}
				\eject
	
		\section{Ostali zahtjevi}
		
			\textbf{\textit{dio 1. revizije}}\\
		 
			 \textit{Nefunkcionalni zahtjevi i zahtjevi domene primjene dopunjuju funkcionalne zahtjeve. Oni opisuju \textbf{kako se sustav treba ponašati} i koja \textbf{ograničenja} treba poštivati (performanse, korisničko iskustvo, pouzdanost, standardi kvalitete, sigurnost...). Primjeri takvih zahtjeva u Vašem projektu mogu biti: podržani jezici korisničkog sučelja, vrijeme odziva, najveći mogući podržani broj korisnika, podržane web/mobilne platforme, razina zaštite (protokoli komunikacije, kriptiranje...)... Svaki takav zahtjev potrebno je navesti u jednoj ili dvije rečenice.}
			 
			 
			 
	