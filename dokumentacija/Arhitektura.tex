\chapter{Arhitektura i dizajn sustava}

		\textit{ Arhitektura ima 3 podsustava:}
	\begin{itemize}
		\item 	\textit{Web preglednik}
		\item 	\textit{Web aplikacija na web poslužitelju}
		\item 	\textit{Baza podataka - PostgreSQL }		
	\end{itemize}


{\underline{Web preglednik} je ujedino i prevoditelj koda koji korisniku omogućuje pregledavanje sadržaja web aplikacije i interakciju s istim.}

{\underline{Web poslužitelj} prima HTTP (engl. Hyper
Text Transfer Protocol) zahtjeve od klijenta (preglednika) koji sadrže informacije o tome što klijent traži, kao što su primjerice URL, GET, POST... i vraća dohvaćeni resurs ako ga ima te vraća statusni kod koji daje informacije o uspješnosti zahtjeva.}

{Tehnologije korištene u našoj aplikaciji temelje se na Spring Bootu i Reactu. Aplikacija se sastoji od serverske komponente napisane u Javi (Spring Boot) i klijentske komponente napisane u JavaScriptu (React). Za razvojno okruženje koristimo IntelliJ, a baza koju koristimo za spremanje podataka o registriranim korisnicima i sve informacije o natjecanjima je PostgrSQL.}

{\underline{Web aplikacija} temelji se na arhitekturi Model-View-Controller (MVC). Ova arhitektura omogućuje organizaciju aplikacije u tri ključne komponente:}

\begin{itemize}
    \item \textbf{Model}: Ova komponenta predstavlja poslovnu logiku i podatke aplikacije. Model je implementiran u Java programskom jeziku (Spring Boot) i odgovoran je za upravljanje podacima, komunikaciju s bazom podataka te izračune i obrade podataka.
    
    \item \textbf{View}: Komponenta za prikaz podataka korisnicima. Prikazi se generiraju u Reactu, a omogućuju korisnicima interakciju s aplikacijom putem web preglednika. Prikazi se oblikuju pomoću HTML-a, CSS-a i JavaScripta.
    
    \item \textbf{Controller}: Kontroler je posrednik između Modela i Viewa. Ova komponenta upravlja korisničkim zahtjevima, prima ulazne podatke od korisnika te izvršava odgovarajuće akcije u Modelu. Kontroler također određuje koji prikaz treba biti poslan korisnicima.\\
\end{itemize}	

				
		\section{Baza podataka}
			

			
		{Koristimo relacijsku bazu podataka čije su gradivne jedinke tablice definirane imenom i skupom atributa za jednostavno upravljanje podacima. Baza podataka ove aplikacije sastoji se od sljedećih entiteta:}
	\begin{itemize}
		\item 	\textit{Korisnik}
		\item 	\textit{Natjecanje}
		\item 	\textit{Zadaci na natjecanju}
		\item 	\textit{Zadatak}				
	\end{itemize}
		
			\subsection{Opis tablica}
			

				{\textbf{Korisnik} - entitet sadržava sve važne informacije o korisniku: Korisničko ime, lozinku, ime, prezime, sliku, email, tip korisnika (natjecatelj/voditelj). }
				
				
				\begin{longtblr}[
					label=none,
					entry=none
					]{
						width = \textwidth,
						colspec={|X[6,l]|X[6, l]|X[20, l]|}, 
						rowhead = 1,
					} %definicija širine tablice, širine stupaca, poravnanje i broja redaka naslova tablice
					\hline \SetCell[c=3]{c}{\textbf{Korisnik}}	 \\ \hline[3pt]
					 \SetCell{LightGreen}id & INT	&   jedinstveni identifikator korisnika	\\ \hline
					 \SetCell{LightBlue} username	& VARCHAR &   	izabrano korisničko ime\\ \hline 
					 \SetCell{LightBlue}email & VARCHAR &  e-mail adresa korisnika \\ \hline 
					 name & VARCHAR	&  	ime korisnika	\\ \hline 
					 lastname & VARCHAR	&  	prezime korisnika	\\ \hline 
					 image & BYTEA	&  	slika korisnika	\\ \hline							user-type & VARCHAR & natjecatelj ili voditelj natjecanja \\ \hline
					 password & VARCHAR	&  	šifra korisnika	\\ \hline 
\end{longtblr}


				{\textbf{Natjecanje} - entitet sadržava sve važne informacije o natjecanju: Id natjecanja, vrijeme početka, vrijeme završetka, broj zadataka, id voditelja natjecanja. U odnosu je Many-To-One s entitetom korisnik preko atributa voditelja natjecanja. }
				
				
				\begin{longtblr}[
					label=none,
					entry=none
					]{
						width = \textwidth,
						colspec={|X[6,l]|X[6, l]|X[20, l]|}, 
						rowhead = 1,
					} %definicija širine tablice, širine stupaca, poravnanje i broja redaka naslova tablice
					\hline \SetCell[c=3]{c}{\textbf{Natjecanje}}	 \\ \hline[3pt]
					 \SetCell{LightGreen}id & INT	&   jedinstveni identifikator natjecanja	\\ \hline
					  date-time-of-beginning	& TIMESTAMP &   vrijeme početka natjecanja	\\ \hline 
					 date-time-of-ending	& TIMESTAMP &   vrijeme završetka natjecanja	\\ \hline  
					 competition-maker-id & INT	&  	id voditelja natjecanja	\\ \hline 
	 number-of-problems & INT	&  	broj zadataka u natjecanju	\\ \hline 
				\end{longtblr}

				{\textbf{Zadatak} - entitet sadržava sve važne informacije o zadatku. Sadrži atribute id, trajanje natjecanja, booleanski atribut is-private, broj bodova koje je moguće ostvariti, tip problema, tekst zadatka, naslov i id korisnika koji je napravio zadatak. S entitetom Korisnik je u odnosu Many-To-One preko atributa problem-maker-id.}
				
		\begin{longtblr}[
					label=none,
					entry=none
					]{
						width = \textwidth,
						colspec={|X[6,l]|X[6, l]|X[20, l]|}, 
						rowhead = 1,
					} %definicija širine tablice, širine stupaca, poravnanje i broja redaka naslova tablice
					\hline \SetCell[c=3]{c}{\textbf{Zadatak}}	 \\ \hline[3pt]
					 \SetCell{LightGreen} id & INT	&   jedinstveni identifikator zadatka	\\ \hline
				  \SetCell{LightBlue}problem-maker-id & INT	& identifikator vlasnika zadatka	\\ \hline 
					 duration &  NUMERIC	& trajanje zadatka	\\ \hline 
					 is-private &  BOOLEAN	& provjerava je li zadatak objavljen	\\ \hline 
					 problem-type &  INT	&  ?	\\ \hline 
					text &  VARCHAR	& tekst zadatka	\\ \hline 
					 title &  NUMERIC	& naslov zadatka	\\ \hline 

				\end{longtblr}

				{\textbf{Zadaci na natjecanju} - entitet sadržava sve važne informacije odnosu zadatka i natjecanja na kojem se nalazi. Atributi: id natjecanja i id zadatka }

				
				\begin{longtblr}[
					label=none,
					entry=none
					]{
						width = \textwidth,
						colspec={|X[6,l]|X[6, l]|X[20, l]|}, 
						rowhead = 1,
					} %definicija širine tablice, širine stupaca, poravnanje i broja redaka naslova tablice
					\hline \SetCell[c=3]{c}{\textbf{Zadaci u natjecanju}}	 \\ \hline[3pt]
					 \SetCell{LightBlue}competition-id & INT	&    identifikator natjecanja (natjecanje.id)	\\ \hline
					 \SetCell{LightBlue} problem-id & INT	& identifikator zadatka	(problem.id) \\ \hline 
				\end{longtblr} 


				

				{\textbf{Input-output mapa} - entitet sadržava id zadatka te vrijednost i ključ, odnosno na ovaj način se mapiraju rješenja zadataka. }

				
				\begin{longtblr}[
					label=none,
					entry=none
					]{
						width = \textwidth,
						colspec={|X[6,l]|X[6, l]|X[20, l]|}, 
						rowhead = 1,
					} %definicija širine tablice, širine stupaca, poravnanje i broja redaka naslova tablice
					\hline \SetCell[c=3]{c}{\textbf{Input-output mapa}}	 \\ \hline[3pt]
					 \SetCell{LightGreen}problem-id & INT	&  jedinstveni identifikator zadatka	\\ \hline
					 vrijednost & VARCHAR	& ?  \\ \hline 
					 ključ & VARCHAR	& ?  \\ \hline 
				\end{longtblr}
				
				
			
			\subsection{Dijagram baze podataka}
				\textit{ U ovom potpoglavlju potrebno je umetnuti dijagram baze podataka. Primarni i strani ključevi moraju biti označeni, a tablice povezane. Bazu podataka je potrebno normalizirati. Podsjetite se kolegija "Baze podataka".}
			
			\eject
			
			
		\section{Dijagram razreda}
		
			\textit{Potrebno je priložiti dijagram razreda s pripadajućim opisom. Zbog preglednosti je moguće dijagram razlomiti na više njih, ali moraju biti grupirani prema sličnim razinama apstrakcije i srodnim funkcionalnostima.}\\
			
			\textbf{\textit{dio 1. revizije}}\\
			
			\textit{Prilikom prve predaje projekta, potrebno je priložiti potpuno razrađen dijagram razreda vezan uz \textbf{generičku funkcionalnost} sustava. Ostale funkcionalnosti trebaju biti idejno razrađene u dijagramu sa sljedećim komponentama: nazivi razreda, nazivi metoda i vrste pristupa metodama (npr. javni, zaštićeni), nazivi atributa razreda, veze i odnosi između razreda.}\\
			
			\textbf{\textit{dio 2. revizije}}\\			
			
			\textit{Prilikom druge predaje projekta dijagram razreda i opisi moraju odgovarati stvarnom stanju implementacije}
			
			
			
			\eject
		
		\section{Dijagram stanja}
			
			
			\textbf{\textit{dio 2. revizije}}\\
			
			\textit{Potrebno je priložiti dijagram stanja i opisati ga. Dovoljan je jedan dijagram stanja koji prikazuje \textbf{značajan dio funkcionalnosti} sustava. Na primjer, stanja korisničkog sučelja i tijek korištenja neke ključne funkcionalnosti jesu značajan dio sustava, a registracija i prijava nisu. }
			
			
			\eject 
		
		\section{Dijagram aktivnosti}
			
			\textbf{\textit{dio 2. revizije}}\\
			
			 \textit{Potrebno je priložiti dijagram aktivnosti s pripadajućim opisom. Dijagram aktivnosti treba prikazivati značajan dio sustava.}
			
			\eject
		\section{Dijagram komponenti}
		
			\textbf{\textit{dio 2. revizije}}\\
		
			 \textit{Potrebno je priložiti dijagram komponenti s pripadajućim opisom. Dijagram komponenti treba prikazivati strukturu cijele aplikacije.}