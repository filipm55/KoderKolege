\chapter{Zaključak i budući rad}
		
		
		Timski rad na projektu Bytepit predstavljao je izazov, ali i priliku za učenje i razvoj. Tijekom vremena izrade projektnog zadatka, suočili smo se s raznim tehničkim izazovima, ali smo istovremeno stekli važna znanja i vještine. U ovom osvrtu, razmatrat ćemo ključne aspekte projekta, uočene izazove, rješenja koja smo primijenili, stečena znanja te preostale mogućnosti za poboljšanja i budući rad.
		
		\textbf{Vrijeme izrade projekta}\\
		Razdoblje izrade projekta bilo je intenzivno, no istovremeno je pružilo priliku za suradnju i učenje. Rad u timu od sedam članova zahtijevao je učinkovitu komunikaciju i koordinaciju, što smo postigli redovitom komunikacijom preko WhatsApp-a te sastancima putem Google Meet-a.
		
		\textbf{Tehnički izazovi}\\
		Tijekom razvoja aplikacije, identificirali smo nekoliko tehničkih izazova, uključujući optimalnu integraciju korisničkog sučelja, upravljanje korisničkim podacima i sigurnosne aspekte vezane uz evaluaciju programskih zadataka te provedbu natjecanja.
		
		\textbf{Stečena znanja}\\
		Izrada projekta Bytepit omogućila nam je široko stjecanje znanja iz područja web razvoja, baza podataka, tehnologija ocjenjivanja programskih zadataka i upravljanja korisnicima. Svaki član tima poboljšao je svoje programerske vještine, stekao iskustvo u radu s tehnologijama poput React i Spring Boot te se upoznao s konceptima sigurnog rukovanja podacima korisnika.
		
		\textbf{Perspektive za nastavak rada}\\
		Daljnji razvoj Bytepita može uključivati proširenje funkcionalnosti, poboljšanje performansi, dodatne sigurnosne aspekte te integracije s drugim platformama recimo Android i IOS.
		
		\textbf{Zaključak}\\
		Projekt Bytepit predstavljao je izazovno, ali izuzetno poučno iskustvo. Kroz suradnju s timom, rješavanje tehničkih izazova te stjecanje raznolikih vještina, svaki član tima doprinio je uspjehu projekta. 
	